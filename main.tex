\documentclass[12pt]{article}
\usepackage[a4paper, margin=1in]{geometry}
\usepackage{titlesec}
\usepackage{graphicx}
\usepackage{fancyhdr}
\usepackage{hyperref}
\usepackage{longtable}
\usepackage{enumitem}
\usepackage{color}
\usepackage{array}
\usepackage{setspace}
\usepackage{caption}

\setlength{\parskip}{0.5em}
\setlength{\parindent}{2em}
\pagestyle{fancy}
\fancyhf{}
\rhead{CPT304 Assignment 2}
\lhead{Open Source Contribution}
\rfoot{Page \thepage}

% Title formatting
\titleformat{\section}{\Large\bfseries}{\thesection}{1em}{}
\titleformat{\subsection}{\normalsize\bfseries}{\thesubsection}{1em}{}

\begin{document}

% -------------------
% Title Page Content
% -------------------
\noindent
\textbf{\Large CPT304 Assignment 2 Report} \\[0.5em]
\textbf{Title:} Open-Source Investigation and Contribution \\[0.5em]
\textbf{Group Members:} 
\\ [0.5em]
\textbf{Date:} \today

\vspace{1em}
\hrule
\vspace{1.5em}

% -------------------
% Abstract
% -------------------
\section*{Abstract}

% This report documents our contribution to the open-source project \texttt{project-name}. It includes an investigation of the open-source community, our team's contribution plan, the implementation process, and the results. We selected a beginner-friendly project and submitted real contributions in the form of bug fixes, documentation updates, and code enhancements.

% -------------------
% 1. Introduction
% -------------------
\section{Introduction}

% The open-source project we selected is \texttt{project-name}, hosted on GitHub. This project is a widely-used tool in the field of data analysis, featuring active maintainers and good onboarding documentation. Our motivation for choosing this project was based on its relevance to our skill sets and the opportunity to make meaningful contributions.

% -------------------------------
% 2. Open Source Investigation
% -------------------------------
\section{Open Source Investigation}

\subsection{Understanding the Open Source Community}
We studied the philosophy behind open source: transparency, collaboration, and community.

\subsection{Open Source Licenses}
The key licenses we investigated include:

\begin{itemize}
    \item \textbf{MIT License}: Highly permissive, allows reuse with attribution.
    \item \textbf{GPLv3}: Requires derivative works to remain open.
    \item \textbf{Apache 2.0}: Includes explicit patent grants.
\end{itemize}

\subsection{Ways to Contribute}
Common contribution types include:
\begin{itemize}
    \item Code (features, bug fixes)
    \item Documentation (guides, tutorials)
    \item Bug reports and issue triage
    \item Community support and moderation
\end{itemize}

\subsection{Finding Suitable Projects}
We explored GitHub, filtering projects with labels such as \texttt{good first issue} and \texttt{help wanted}. Tools like \texttt{firstcontributions.github.io} were also helpful.

\subsection{Typical Contribution Process}
The typical process includes:
\begin{enumerate}
    \item Fork the repository
    \item Clone to local machine
    \item Create a feature branch
    \item Commit and push changes
    \item Open a pull request
\end{enumerate}

% ----------------------
% 3. Contribution Plan
% ----------------------
\section{Contribution Plan}

Our group has decided that the team leader will be responsible for coordinating the documentation plan and scheduling, as well as facilitating thorough discussions with all members to select an open-source project for contribution. Each team member will make contributions through their own local repositories after forking the selected project, and a single pull request will be submitted on behalf of the entire group. The specific division of tasks among our members is as follows.

\begin{longtable}{|p{3cm}|p{7cm}|p{3cm}|}
\hline
\textbf{Name} & \textbf{Task} & \textbf{Status} \\
\hline
 & Coordinate scheduling, write abstracts/plans/conclusions, final layout, final contribution & Pending \\
\hline
 & License, community ecosystem, contribution process description & Pending \\
\hline
 & sample & Pending \\
\hline
 & sample & Pending \\
\hline
\end{longtable}

% ----------------------
% 4. Implementation
% ----------------------
\section{Implementation}

We followed a collaborative workflow using GitHub:

\begin{itemize}
    \item Forked the upstream repository
    \item Cloned and created local branches
    \item Used VS Code, pytest, and Markdown tools for implementation
    \item Opened pull requests with detailed descriptions and commit messages
\end{itemize}

% ----------------------
% 5. Results
% ----------------------
\section{Results}

Our contributions include:
% \begin{itemize}
%     \item Pull Request \#101: Bug fix in `api.py` — \url{https://github.com/example/project/pull/101}
%     \item Pull Request \#102: Improved documentation — \url{https://github.com/example/project/pull/102}
% \end{itemize}

% \begin{figure}[h!]
%     \centering
%     % \includegraphics[width=0.85\textwidth]{pr_screenshot.png}
%     \caption{Screenshot of pull request submitted and merged}
% \end{figure}

% ----------------------
% 6. Discussion
% ----------------------
\section{Discussion}

Contributing to a real-world project gave us first-hand experience with open-source collaboration. We received valuable feedback from the maintainer. One PR was merged within 24 hours, while another received comments suggesting improvements. We learned the importance of clear communication, readable code, and community etiquette.

% ----------------------
% 7. Conclusion
% ----------------------
\section{Conclusion}
In this project, our team first collaborated to complete a survey of open source contributions, including the significance of open source and how to become an open source contributor. Then we selected an actual meaningful open source project, and actually checked and contributed to the project. Through this course, we learned Git skills and the standard contribution process, and learned how to use git for team collaboration. We plan to continue to pay attention to open source projects outside of this course and continue to contribute.


% ----------------------
% References
% ----------------------
\section*{References}
\begin{itemize}
    \item GitHub Docs: \url{https://docs.github.com/}
    \item First Contributions Guide: \url{https://firstcontributions.github.io}
    \item Open Source Guides: \url{https://opensource.guide/}
\end{itemize}

% ----------------------
% Appendix
% ----------------------
\newpage
\appendix
\section*{Appendix}

\begin{center}
    \LARGE \textbf{CPT304 Assignment 2} \\
    \vspace{0.5em}
    \large \textbf{Individual Contribution}
\end{center}

\vspace{2em}

\noindent\textbf{Group Number:}

\vspace{1.5em}

\renewcommand{\arraystretch}{1.8}
\begin{tabular}{|>{\arraybackslash}m{5cm}|>{\centering\arraybackslash}m{4cm}|>{\centering\arraybackslash}m{3cm}|}
\hline
\textbf{Name} & \textbf{ID Number} & \textbf{Contribution (\%)} \\
\hline
1. &  &  \\
\hline
2. &  &  \\
\hline
3. &  &  \\
\hline
4. &  &  \\
\hline
\end{tabular}

\vspace{3em}

\noindent\textbf{Signed by all members:}

\vspace{1em}

\noindent\rule{16cm}{0.4pt}

\end{document}