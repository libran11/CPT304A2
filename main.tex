\documentclass[12pt]{article}
\usepackage[a4paper, margin=1in]{geometry}
\usepackage{titlesec}
\usepackage{graphicx}
\usepackage{fancyhdr}
\usepackage{hyperref}
\usepackage{longtable}
\usepackage{enumitem}
\usepackage{color}
\usepackage{array}
\usepackage{setspace}
\usepackage{caption}

\setlength{\parskip}{0.5em}
\setlength{\parindent}{2em}
\pagestyle{fancy}
\fancyhf{}
\rhead{CPT304 Assignment 2}
\lhead{Open Source Contribution}
\rfoot{Page \thepage}

% Title formatting
\titleformat{\section}{\Large\bfseries}{\thesection}{1em}{}
\titleformat{\subsection}{\normalsize\bfseries}{\thesubsection}{1em}{}

\begin{document}

% -------------------
% Title Page Content
% -------------------
\noindent
\textbf{\Large CPT304 Assignment 2 Report} \\[0.5em]
\textbf{Title:} Open-Source Investigation and Contribution \\[0.5em]
\textbf{Group Members:} 
\\ [0.5em]
\textbf{Date:} \today

\vspace{1em}
\hrule
\vspace{1.5em}

% -------------------
% Abstract
% -------------------
\section*{Abstract}
The open-source ecosystem is the innovation driver of modern software engineering, and learning about open-source values and practices is of great significance to developers. This course report presents our team's analysis of the entire process from contributors' workflows to the acceptance of contributions, with actual contributions made in the open-source project \texttt{Python-Scripts}. 

We first investigated the meaning and value of open source, elaborating on different types of open-source licenses, followed by an analysis of governance models and contribution processes within the open-source community. Based on this research, we explored real-world open-source processes through practice. In our contribution plan, all four team members collaborated by enhancing code and documentation for the open-source project while dividing tasks related to this report's content. 

We submitted a pull request to the \texttt{Python-Scripts} repository, contributing beginner-friendly Python scripts with practical functionality and improving the project’s documentation. Our contributions are currently under review by the maintainers. Through this process, we demonstrated effective teamwork and gained valuable experience in open-source collaboration.

% -------------------
% 1. Introduction
% -------------------
\section{Introduction}
Open source software is a pillar of modern engineering practice, providing publicly auditable codebases and an active peer review process. Contributing to the open source community allows developers to gain practical experience in version control, issue categorization, continuous integration, and distributed collaboration. In this report, our team selected the \texttt{Python-Scripts} project. 

\texttt{Python-Scripts} is a community-maintained repository that brings together over 60 independent Python tools, each located in its own folder with explanatory README files. The project uses the MIT license, has over 700 GitHub stars, and is tagged with labels suitable for beginners (such as "good-first-issue"). 

We chose it for three reasons: 
\begin{itemize}
    \item \textbf{Technical scope fit:} The repository covers file input/output, web automation, and simple machine learning demonstrations—technologies we have encountered in our course.
    \item \textbf{Low entry barrier:} The modular folder structure allows newcomers to add or optimize individual scripts without deep coupling with the rest of the codebase.
    \item \textbf{Responsive maintainers:} Historical PR data shows shorter review cycles which significantly increases the likelihood of our contributions being merged and available for evaluation verification.
\end{itemize}


% -------------------------------
% 2. Open Source Investigation
% -------------------------------
\section{Open Source Investigation}

\subsection{Understanding the Open Source Community}
We studied the philosophy behind open source: transparency, collaboration, and community. The core of the open-source movement lies in the principle of "sharing is progress." When source code is made publicly available, anyone can read, improve, and redistribute it, which promotes a highly transparent collaborative culture on a global scale. 

Developers are no longer isolated islands but aggregate into efficient communities through mailing lists, Issues discussion areas, and instant messaging tools. For newcomers, the most direct benefit is the quick acquisition of mature code and learning best practices; for enterprises, open-source means lower cost of trial and error and higher speed of innovation; for us students, we can learn coding standards in actual production, and make connections with peers, receive code reviews, and learn best practices through the open-source community. 

The open-source community usually follows a governance model of consensus-driven and gatekeeping by maintainers: the majority opinion guides the technical direction, but the power to merge code ultimately rests with a few core maintainers.

\subsection{Open Source Licenses}
The key licenses we investigated include:

\begin{itemize}
    \item \textbf{MIT License}: Highly permissive, allows reuse with attribution.
    \item \textbf{GPL v3}: Requires derivative works to remain open.
    \item \textbf{Apache 2.0}: Includes explicit patent grants.
\end{itemize}
    The following is a detailed introduction to the licenses we researched.
    
    \noindent\textbf{MIT License}

    The MIT License is one of the simplest and most permissive open-source licenses currently available. It allows anyone to use, copy, modify, and publish code with almost no restrictions, including commercializing it. Developers only need to retain the original copyright notice and license text when distributing the source code to use it legally. The license explicitly states that the software is provided "as is," without any form of warranty liability. The author assumes no responsibility for issues related to functionality, security, or suitability. This provides developers with great flexibility while reducing legal risks. However, the MIT License does not cover patent rights nor does it provide guidance on trademark usage. Therefore, compliance should be assessed when the code involves significant algorithms or trademark usage. Due to its high compatibility, this license is widely used in front-end frameworks (such as React and Vue) as well as various scaffolding tools and lightweight general-purpose libraries.

    \noindent\textbf{GNU General Public License v3 (GPL v3)}

    GPL v3 is a strong Copyleft open-source license released by the Free Software Foundation in 2007. The agreement requires that any derivative works built on GPL v3 code must be made available under the same license whenever they are distributed, ensuring that software freedom can be inherited and continued. In addition to inheritance requirements, GPL v3 also prohibits preventing users from freely using or modifying the software through DRM technologies or hardware restrictions. Furthermore, the agreement includes an automatic patent grant clause, whereby contributors automatically grant their patent licenses related to their code to users. Moreover, if a user who uses GPL v3 licensed code initiates a patent lawsuit against the author, their originally granted patent rights will also be immediately terminated. This agreement is widely used in free software projects such as the GNU toolchain, GDB, Bash, and other core components of operating systems and is a representative license advocating for software freedom.


  \noindent\textbf{Apache License 2.0}

    The Apache License 2.0, released by the Apache Software Foundation, is a permissive yet more mature open-source license. It allows users to freely use, modify, and distribute code while further adding provisions for patent rights protection, making it widely adopted in enterprise-level projects. The license explicitly states that contributors automatically grant users a global, irrevocable, royalty-free patent license related to their contributions. Additionally, it introduces a patent retaliation termination mechanism: if a user initiates a patent lawsuit against the original author due to the use of this code, their licensing rights will be immediately revoked. To ensure compliance, Apache 2.0 requires that LICENSE and NOTICE files be retained during distribution to document changes and third-party components relied upon. This plays an important role in corporate governance and compliance audits. Apache 2.0 is compatible with GPL v3 but not with GPL v2 and is currently the mainstream choice in cloud computing and machine learning fields. Notable open-source projects such as Hadoop, Kafka, Spark, TensorFlow all utilize this license.
% \vspace{1em}

\noindent\textbf{Comparison}

The \textbf{MIT License} is highly compatible and has lenient terms, making it very suitable for developers who want to quickly promote projects, attract more developers to participate, and even consider commercial closed-source integration in the future. It is most common in the fields of front-end frameworks, development tools, lightweight libraries, and other areas. 

The \textbf{Apache 2.0 License}, while maintaining the flexibility of open source, provides patent authorization and anti-suit protection mechanisms, reducing enterprises' concerns about compliance. Enterprise-level infrastructure projects such as cloud computing components, distributed systems, middleware, and machine learning frameworks mainly choose this open source protocol. If the code is for enterprise services or if you want to integrate with the ecosystem of large companies, Apache 2.0 is a more suitable choice. 

The \textbf{GPL v3 License} emphasizes that all derivative works must remain open source. General toolchains, development platforms, educational public projects, and other projects that need to ensure that the code remains open mainly choose this protocol. If developers value software freedom and do not want others to use your code in a closed-source manner, then choosing GPL v3 is appropriate.


\subsection{Ways to Contribute}
Common contribution types include:
\begin{itemize}
    \item Code (features, bug fixes)
    \item Documentation (guides, tutorials)
    \item Bug reports and issue triage
    \item Community support and moderation
\end{itemize}

\subsection{Finding Suitable Projects}
We explored GitHub, filtering projects with labels such as \texttt{good first issue} and \texttt{help wanted}. Tools like \texttt{firstcontributions.github.io} were also helpful.

\subsection{Typical Contribution Process}
The typical process includes:
\begin{enumerate}
    \item Fork the repository
    \item Clone to local machine
    \item Create a feature branch
    \item Commit and push changes
    \item Open a pull request
\end{enumerate}

\subsection{Contribute to \texttt{Python-Scripts}}


% ----------------------
% 3. Contribution Plan
% ----------------------
\section{Contribution Plan}

Our group has decided that the team leader will be responsible for coordinating the documentation plan and scheduling, as well as facilitating thorough discussions with all members to select an open-source project for contribution. Each team member will make contributions through their own local repositories after forking the selected project, and a single pull request will be submitted on behalf of the entire group. The specific division of tasks among our members is as follows.

\begin{longtable}{|p{3cm}|p{7cm}|p{3cm}|}
\hline
\textbf{Name} & \textbf{Task} & \textbf{Status} \\
\hline
 & Coordinate scheduling, write abstracts/plans/conclusions, final layout, final contribution & Pending \\
\hline
 & License, community ecosystem, contribution process description & Pending \\
\hline
 & sample & Pending \\
\hline
 & sample & Pending \\
\hline
\end{longtable}

% ----------------------
% 4. Implementation
% ----------------------
\section{Implementation}

We followed a collaborative workflow using GitHub:

\begin{itemize}
    \item Forked the upstream repository
    \item Cloned and created local branches
    \item Used VS Code, pytest, and Markdown tools for implementation
    \item Opened pull requests with detailed descriptions and commit messages
\end{itemize}

% ----------------------
% 5. Results
% ----------------------
\section{Results}

Our contributions include:
% \begin{itemize}
%     \item Pull Request \#101: Bug fix in `api.py` — \url{https://github.com/example/project/pull/101}
%     \item Pull Request \#102: Improved documentation — \url{https://github.com/example/project/pull/102}
% \end{itemize}

% \begin{figure}[h!]
%     \centering
%     % \includegraphics[width=0.85\textwidth]{pr_screenshot.png}
%     \caption{Screenshot of pull request submitted and merged}
% \end{figure}

% ----------------------
% 6. Discussion
% ----------------------
\section{Discussion}

Contributing to a real-world project gave us first-hand experience with open-source collaboration. We received valuable feedback from the maintainer. One PR was merged within 24 hours, while another received comments suggesting improvements. We learned the importance of clear communication, readable code, and community etiquette.

% ----------------------
% 7. Conclusion
% ----------------------
\section{Conclusion}
In this project, our team first collaborated to complete a survey of open source contributions, including the significance of open source and how to become an open source contributor. Then we selected an actual meaningful open source project \texttt{Python-Scripts}, and actually contributed to the project. Through this course, we learned Git skills and the standard contribution process, and learned how to use git for team collaboration. We plan to continue to pay attention to open source projects outside of this course and continue to contribute.


% ----------------------
% References
% ----------------------
\section*{References}
\begin{itemize}
    \item \texttt{Python-Scripts} GitHub Repo: \url{https://github.com/DhanushNehru/Python-Scripts}
    \item GitHub Docs: \url{https://docs.github.com/}
    \item First Contributions Guide: \url{https://firstcontributions.github.io}
    \item Open Source Guides: \url{https://opensource.guide/}
\end{itemize}

% ----------------------
% Appendix
% ----------------------
\newpage
\appendix
\section*{Appendix}

\begin{center}
    \LARGE \textbf{CPT304 Assignment 2} \\
    \vspace{0.5em}
    \large \textbf{Individual Contribution}
\end{center}

\vspace{2em}

\noindent\textbf{Group Number:}

\vspace{1.5em}

\renewcommand{\arraystretch}{1.8}
\begin{tabular}{|>{\arraybackslash}m{5cm}|>{\centering\arraybackslash}m{4cm}|>{\centering\arraybackslash}m{3cm}|}
\hline
\textbf{Name} & \textbf{ID Number} & \textbf{Contribution (\%)} \\
\hline
1. &  &  \\
\hline
2. &  &  \\
\hline
3. &  &  \\
\hline
4. &  &  \\
\hline
\end{tabular}

\vspace{3em}

\noindent\textbf{Signed by all members:}

\vspace{1em}

\noindent\rule{16cm}{0.4pt}

\end{document}